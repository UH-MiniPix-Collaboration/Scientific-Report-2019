\subsection{Astrobiology Results}
\label{sec:Astrobiology-Results}

The SORA astrobiology team is currently still waiting for the biological samples collected during the flight to be sequenced and analyzed by Dr. Preethi Gunaratne’s biochemical research laboratory at the University of Houston. Despite this lack of data, conclusions may still be drawn with respect to the success of this year’s mission. During the flight, the astrobiology collection system experienced mechanical difficulties that greatly damaged the integrity of the samples collected. The cameras on the payload showed that, while the collection system had successfully deployed, the rotating collection arms had stopped mid-turn. This situation, while not ideal, would have still proven workable if not for the fact that this also meant that the lid of the collection system was unable to close, leaving the stratospheric samples exposed to contamination. All attempts to resume the rotation of the system were unsuccessful, and while the option of attempting to close the box anyway was considered it was ultimately decided that the risk of the lid popping off and becoming a dangerous projectile was too high. Thus, the collection system box was open and exposed during descent and landing, and remained so until the box was retrieved. Despite the compromised state of the samples, it was decided that the samples would still be sequenced to see if any useful data could be obtained to inform future missions.				

A typical DNA extraction kit was used to extract and purify the DNA found in both the control and experimental samples. The extraction separated the DNA in the samples from other biological molecules (proteins, lipids, etc.) of no interest, making the samples suitable for sequencing. All members of the astrobiology team participated in the extraction, under the supervision of Dr. Donna Pattison, a faculty member of the University of Houston Department of Biology and Biochemistry. The prepared samples were then sent to Dr. Gunaratne’s lab to await sequencing. The sequencing procedure will be very similar to those of the two previous missions, with some notable differences. In the sequencing of the samples obtained from the 2018 mission, it was found that the data was not consistent with that of the 2017 mission—most notably, there was an absence of archaea, a domain of life encompassing organisms that resemble bacteria and are known for their ability to survive in extreme conditions. However, after further investigation, it was found that improper primers were used in the 2018 analysis, making it very possible that data regarding archaea was lost. Primers, which amplify the signal of the DNA being analyzed to detectable levels, are very particular to the class of the organism being studied; thus, picking primers appropriate to the situation is vital. With this to consider, it has been assured that the Gunaratne lab will use primers that will apply to a broader range of organisms, including archaea. Despite the contamination of the samples, it is hoped that the data obtained will confirm some of our previous findings, and provide a better starting point for any future missions. 