\subsection{Astrobiology Methods}
\label{sec:Astrobiology-Methods}

\subsubsection{Construction \& Sanitation}
\label{sec:Construction and Sanitation}
The parts were designed in CAD and machined at the University of Houston College of Natural Sciences and Mathematics Workshop. The material used to construct the pieces was a semi-transparent, polycarbonate material with high hardness rating and can withstand a wide temperature range without significant deformation. The clean box itself, including the lid, was constructed using six plates of material with very tight tolerances. 

The sterilization procedure was a multistep process requiring different methods for various parts of the
collection system. The collection box, control boxes, and various tools for assembly were placed in bags
and run through an autoclave, which is a high pressure and temperature oven designed to sterilize items in
biology labs. It was necessary to perform this procedure without the motors or filters installed, as they
would be damaged by the conditions of the autoclave. The filters were sterilized by placing them in a
clean box and exposing them to UV light for 30 minutes on each side. Once this was done, the boxes and
tools were removed from the autoclave and placed in the clean box for assembly, after thoroughly wiping
down the bags they were autoclaved in with a 70\% ethanol solution. Every individual who placed their
hands in the clean box wore nitrile gloves, and rinsed their hands with the ethanol solution each time they
entered the clean box. The collection system was then assembled, with everything that could not be
autoclaved or placed under UV light—the motors, epoxy, etc.—being thoroughly wiped down with the
ethanol solution before being placed in the clean box. Once assembly was complete, the control boxes and
the collection box were sealed, ensuring that the interiors remained sterile until deployment and analysis.

Extensive sanitation procedure was carried out before assembly to eliminate any possible sources of contaminating bacteria. The entire collection structure, except for the L-16 linear servo motor and the Fluropore membrane filters, along with any tools used for construction were thermally sterilized in an autoclave at \SI{120}{\celsius} for 50 minutes. Then, the servo was wiped down with 70\% ethanol solution, and the Fluropore filters were exposed to intense UV light for several hours before assembly.

\subsubsection{Assembly \& transportation}
\label{sec:Assembly}
The assembly process was performed within a SterilGARD e3 Class II Biological Safety Cabinet. Every personnel involved in assembly was garbed in a lab coat, goggles, hair net and latex gloves after thoroughly washing their hands in a 70\% ethanol solution. The filters were threaded onto the wings, and the collection mechanism consisting of the servos and the mechanical arms were installed in the clean box. Gaskets were placed along the edges where the boxes come into contact with the lids and secured in place with vacuum epoxy. In addition, the screws of the box, as well as any gaps between the plates, were also covered with epoxy to create an airtight seal. 

After the assembly process, a signal was sent to the motor to close the lid and seal off the clean box to prepare for transportation. After the collection module was integrated into the rest of the payload, the entire payload will be placed in an autoclave bag to transport to the flight site.

