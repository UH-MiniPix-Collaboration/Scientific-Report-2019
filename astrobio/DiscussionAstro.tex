\subsection{Astrobiology Discussion}
\label{sec:Astrobiology-Discussion}

During the flight, the astrobiology collection system experienced mechanical difficulties that greatly damaged the integrity of the samples collected. The cameras on the payload showed that, while the collection system had successfully deployed, the rotating collection arms had stopped mid-turn. This situation, while not ideal, would have still proven workable if not for the fact that this also meant that the lid of the collection system was unable to close, leaving the stratospheric samples exposed to contamination. All attempts to resume the rotation of the system were unsuccessful, and while the option of attempting to close the box was considered it was ultimately decided that the risk of the lid popping off and becoming a dangerous projectile was too high. Thus, the collection system box was open and exposed during descent and landing, and remained so until the box was retrieved. Despite the compromised state of the samples, it was decided that the samples would still be sequenced to see if any useful data could be obtained to inform future missions. This error could be solved with a circular design for the lid as that would prevent the lid from nto beign able to close properly.
