\subsection{Radiation Background}
\label{sec:Radiation-Background}

Primary comsic rays is an umbrella term that describes a group of high-energy, charged particles.
These particles are typically partially or fully ionized atoms with the major constituents being hydrogen and helium nuclei respectively making up about 89\% and 10\% of the spectrum \cite{Lemoine} \cite{CERN-Cosmic-Rays}.
When a primary particle collides with a molecule in the atmosphere, secondary particles are produced which then collide with molecules causing a cascade of particles known as an air shower \cite{Frank-Schroder}.
The products of air showers are electrons, muons, neutrinos, electromagnetic waves, and various other particles which can be detected on the Earth's surface.
By measuring these particles and gathering information about the air shower, the type, energy, and direction of the primary particle can be extrapolated.
The initial energy of the primary particle determines the penetration depth of the air shower, however, most air showers have a maximum secondary particle density at an altitude within the range of \SIrange{15}{20}{\kilo\meter} known as the Regener-Pfotzer Maximum \cite{Regener-Pfotzer} \cite{Secondary-Intensity-Balloons}.
%In extreme cases when the primary particle is in the ultra-high energy range, the shower maximum can be as low as \SI{3}{\kilo\meter} above the Earth's surface \cite{Frank-Schroder}.
The altitude and strength of this maximum is dependent on several factors such as the solar cycles, time of year, and latitude and longitude.
The cosmic ray flux has contributions from the Sun (solar cosmic rays), the galaxy, and outside the galaxy \cite{Frank-Schroder}.
Cosmic rays which have galactic and extra-galactic origins are known as galactic cosmic rays (GCRs) and are responsible for the majority of the observed particle flux in the atmosphere.
Interestingly, the GCR flux modulates inversely with the solar cycle in events that are known as a Forbush Decrease \cite{Secondary-Intensity-Balloons} \cite{Hathaway-Solar-Cycle} \cite{Forbush-Decrease}.
NASA reports that astronauts aboard the International Space Station (ISS) receive about double the radiation dose during a solar minimum than they do during a solar maximum \cite{NASA-Radiation-Book}.
Furthermore, recent behavior exhibited by the Sun has resulted in the highest observed GCR flux since the dawn of the space age \cite{Schwadron}, causing concern regarding the safety of humans in space.

Naturally, this unexpected behavior from the Sun calls for careful monitoring of humans in space and the skies.
Commercial airplanes fly in a portion of the atmosphere in which the radiation flux is higher, resulting in increased radiation exposure for frequent fliers and airline crew compared to those who fly at most a couple times a year. 
This is a huge motivation for the radiation monitoring portion of our experiment with which we aim to develop a nonintrusive, inexpensive radiation dosimeter.
SORA uses Medipix-based \cite{Medipix} devices to record data.
The SORA 3 experiment used two such devices: one is the MiniPIX \cite{MiniPIX-ADVACAM}, and the other is a FITPix \cite{FITPix}.
These devices have a USB interface and are capable of detecting charged particles by use of a Timepix sensor \cite{Timepix}.
The Timepix sensor is made up of an array of 256 by 256 pixels with a total surface area of \SI{198.25}{\milli\meter\squared}.
Each pixel in the array is its own sensor with its own set of readout electronics.
The details of the chip construction is given by Jakubek in Ref. \cite{Jakubek-Pixel-Detectors}.

The previous two SORA experiments possessed solely the MiniPIX sensor in very similar configurations each time.
SORA 3 expanded upon this construction with the addition of the FITPix in attempt to accurately model the environment inside the ISS with a small container.
Ideally, this model would be effective, enabling data collection that is comparable to that data collected aboard the ISS.
If successful, this would prove the effectiveness of relatively simple constructions to model rather complex or exotic environments.
Furthermore, this would urge more accurate modelling of environments in which human lives are at great risk.
Lastly, this would push the boundary of using Timepix devices as radiation dosimeters.

	For the third University of Houston's (UH) third mission with the HASP program, we elected to include a new research project along with our prior two projects. Following the success of our previous two missions \cite{SORA1} \cite{SORA2} using the semiconductor based MiniPIX radiation detector we decided to investigate the performance of solar cells in the stratosphere. For space applications \cite{space power}, solar panels are judged on their deliverable power divided by their weight, known as specific power [\si{\watt\per\kilo\gram}], and their deliverable power divided by the stowed packing volume, known as the stowed packing efficiency [\si{\watt\per\volt}]. When considering these two critical factors we determined that organic photovoltaic (OPV) materials have the potential to maximize both values\cite{OPV space}, contingent on their successful operation in the near space environment. This decision is based on the fact that the weight of OPV modules is nearly completely determine by the weight of the substrate on which it is processed, and if that substrate is a flexible foil it allows the stowed packing volume to be cut down by multiple orders of magnitude along with the weight when compared to traditional Gallium Arsenide (GaAs) or Silicon (Si) panels. One of the drawbacks of OPV has been lower efficiencies when compared to other PV technologies, however we feel this is not a disadvantage for the satellite and LEO spacecraft industry where energy demands are not as great.\\
	
	 Our group ran into delays finding a functioning lab space and in achieving efficiencies $>$3\%. Originally we had planned on analyzing both polymeric OPVs as well as perovskite based hybrid-OPV modules, but focused primarily on the polymer OPV modules with the understanding that perovskite cells are fabricated in a similar way\cite{perovskite} and have been achieving much high conversion efficiencies than the polymer modules. For this reason, we consider this year's mission to be a proof of concept. A circuit was developed by the electronics team to analyze each cell during flight, and the radiation team was able to fabricate OPV modules in a UH lab. We hope to continue our investigation in the future by considering modules with greater conversion efficiencies or more complex architectures (such as tandem cells) which require more complex fabrications. 
