\subsection{Radiation Background}
\label{sec:Radiation-Background}

Primary comsic rays is an umbrella term that describes a group of high-energy, charged particles.
These particles are typically partially or fully ionized atoms with the major constituents being hydrogen and helium nuclei respectively making up about 89\% and 10\% of the spectrum \cite{Lemoine} \cite{CERN-Cosmic-Rays}.
When a primary particle collides with a molecule in the atmosphere, secondary particles are produced which then collide with molecules causing a cascade of particles known as an air shower \cite{Frank-Schroder}.
The products of air showers are electrons, muons, neutrinos, electromagnetic waves, and various other particles which can be detected on the Earth's surface.
By measuring these particles and gathering information about the air shower, the type, energy, and direction of the primary particle can be extrapolated.
The initial energy of the primary particle determines the penetration depth of the air shower, however, most air showers have a maximum secondary particle density at an altitude within the range of \SIrange{15}{20}{\kilo\meter} known as the Regener-Pfotzer Maximum \cite{Regener-Pfotzer} \cite{Secondary-Intensity-Balloons}.
%In extreme cases when the primary particle is in the ultra-high energy range, the shower maximum can be as low as \SI{3}{\kilo\meter} above the Earth's surface \cite{Frank-Schroder}.
The altitude and strength of this maximum is dependent on several factors such as the solar cycles, time of year, and latitude and longitude.
The cosmic ray flux has contributions from the Sun (solar cosmic rays), the galaxy, and outside the galaxy \cite{Frank-Schroder}.
Cosmic rays which have galactic and extra-galactic origins are known as galactic cosmic rays (GCRs) and are responsible for the majority of the observed particle flux in the atmosphere.
Interestingly, the GCR flux modulates inversely with the solar cycle in events that are known as a Forbush Decrease \cite{Secondary-Intensity-Balloons} \cite{Hathaway-Solar-Cycle} \cite{Forbush-Decrease}.
NASA reports that astronauts aboard the International Space Station (ISS) receive about double the radiation dose during a solar minimum than they do during a solar maximum \cite{NASA-Radiation-Book}.
Furthermore, recent behavior exhibited by the Sun has resulted in the highest observed GCR flux since the dawn of the space age \cite{Schwadron}, causing concern regarding the safety of humans in space.

Naturally, this unexpected behavior from the Sun calls for careful monitoring of humans in space and the skies.
Commercial airplanes fly in a portion of the atmosphere in which the radiation flux is higher, resulting in a radiation exposure for frequent fliers and airline crew. 
This is a huge motivation for the radiation monitoring portion of our experiment with which we aim to develop a nonintrusive, inexpensive radiation dosimeter.
SORA uses Medipix-based \cite{Medipix} devices to record data.
The SORA 3 experiment used two such devices: one was the MiniPIX \cite{MiniPIX-ADVACAM}, and the other was a FITPix \cite{FITPix}.
These devices have a USB interface and are capable of detecting charged particles by use of a Timepix sensor \cite{Timepix}.
The Timepix sensor is made up of an array of 256 by 256 pixels with a total surface area of \SI{198.25}{\milli\meter\squared}.
Each pixel in the array is its own sensor with its own set of readout electronics.
The details of the chip construction is given by Jakubek in Ref. \cite{Jakubek-Pixel-Detectors}.

%While not investigated in our experiment, it is worthwhile to mention the relevance of neutrons when considering radiation dose.
%<<FINISH THIS>>
