\subsection{Cosmic Radiation Methods}
\label{sec:Cosmic-Radiation-Methods}

The MiniPIX has a built-in temperature sensor that can report the temperature of the device, so this function was used to monitor its temperature.
Unfortunately the FITPix does not possess this functionality, but considering the container which housed the FITPix has been tested in previous flights, there was no worry of the FITPix overheating.

The Timepix sensors output data in a human-readable, array-like format that directly corresponds to the pixel array of the sensor.
The sensor collects data in a manner similar to that of an optical camera.
A shutter time is a parameter defined by the user, and the sensor will collect data for the entire duration of the shutter time.
Once the time has lapsed, the sensor packages the data into a data frame and is sent to the storage of the computer.
The length of the shutter time must be carefully chosen, and the time for the SORA 3 mission was approximately 1 second for both devices.
Too short of a length will result in many empty data frames that waste storage space, but too long of a length will result in data frames which are unreadable since the particle tracks in the frame will be indistinguishable from one another.
Each pixel in the 256 by 256 sensor array has one corresponding energy value per frame, so this array can be used to reconstruct the data frame.
The energy threshold is another user-defined parameter that determines the minimum energy which can be detected by the sensor.
The primary purpose of the energy threshold is to filter out extraneous whitenoise from the environment.
An example of a data frame can be seen in Figure \ref{fig:minipix-example-frame}
A separate file contains the metadata for each data frame, and contained within the metadata is information such as the energy threshold and the timestamp.

\begin{figure}[h!]
	\begin{center}
		\includegraphics[width=\textwidth]{figures/interesting-frame-3447.png}
		\caption{Example of a Timepix data frame. This is one of the MiniPIX data frames from the 2019 mission.}
		\label{fig:minipix-example-frame}
	\end{center}
\end{figure}

% Detail the analysis techniques used to determine the morphology and the dose (see 2018 report)

%To analyze the Timepix output, a couple of parameters regarding the sensor itself are required.
Using the techniques detailed by Jakubek et. al. \cite{Jakubek-Analysis}, the radiation dose deposited in the sensor by a particle can be calculated using the following equation
\begin{equation*}
  D_{Si} = \dfrac{E}{M_d},
\end{equation*}
where $E$ is the total energy that a particle deposited into the sensor and $M_d$ is the mass of the sensor.
It is important to note that this dose value is not the same as dose deposited in human flesh, which is referred to as dose equivalent.
Conventional calculations for dose equivalent in tissue requires use of Monte Carlo simulations to calculate a conversion factor \cite{Stuart-Thesis} and is outside the SORA missions' scope of study.
%This computation prevents the real-time calculation of dose that was performed in the SORA missions. 
Calculating the total energy of a particle requires the use of a clustering algorithm.
The clustering algorithm analyzes the raw matrix output from the Timepix sensor and groups the pixels with non-zero energy into clusters, or tracks.
The total energy of the cluster, $E$, is the sum of all pixel energies in the cluster.

%This analysis method is relatively simple to understand, but the techniques required to compute the clusters is rather complex.
Another property of interest from the Timepix data is the linear energy transfer (LET), which provides a standard with which all particle clusters can be examined independent of the particle's incident angle.
%This is particularly important with Timepix sensors since the particle's zenith angle cannot accurately be determined. 
To calculate the LET in the silicon sensor, the following relation is used
\begin{equation*}
  LET_{Si} = \dfrac{E}{L},
\end{equation*}
where $E$ is the same energy value used in the $D_{Si}$ calculation, and $L$ is the length of the track in the detector in three dimensions.
The clustering algorithm uses a typical flood-fill technique to group touching pixels with non-zero energy.
A minimum-area bounding box is then constructed around the cluster with a linear least square fit line which intersects the bounding box, and the projected track length $L_p$ is taken to be the length of the linear least square fit line.
A visual example of the bounding box and the linear least square fit line can be seen seen in Figure \ref{fig:stuart-track-example}.
\begin{figure}[h!]
	\begin{center}
		\includegraphics[width=0.75\textwidth]{figures/stuart-track-example.png}
		\caption{A visual from Ref. \cite{Stuart-Thesis} which shows the minimum-area bounding box (the dotted black box surrounding the particle track) and the linear least square fit line (the solid black line running along the particle track).}
		\label{fig:stuart-track-example}
	\end{center}
\end{figure}
$L$ can now be calculated using $L_p$ and $T$, the thickness of the sensor with
\begin{equation*}
  L = \sqrt{L_p^2 + T^2}.
\end{equation*}
Since $T$ is a known property of the sensor, the LET value can then be calculated on a per particle basis.
While out of the scope of the SORA 3 studies, the LET can be used to calculate the dose equivalent in human flesh.

The primary method of categorizing the particle incident on the detector is by the morphology of the resulting cluster.
The track seen in the detector data will change its shape depending on the energy and the type of the particle.
The actual identity of the incident particle cannot be directly measured by the sensor, but an inference can be made by using the energy and shape of the cluster. Ref. \cite{Stuart-Thesis} gives some examples, stating that heavy tracks often correspond to high energy protons and alpha particles, medium blobs can correspond to very slow charged particles, straight tracks can be caused by light minimum ionizing particles (e.g. muon, pion). and light tracks can be caused by electrons and positrons.     
Figure \ref{fig:stuart-track-types} shows the classifications used in the SORA experiments.
\begin{figure}[h!]
	\begin{center}
		\includegraphics[width=\textwidth]{figures/stuart-track-types.pdf}
		\caption{A visual from Ref. \cite{Stuart-Thesis} showing the categories of tracks. This classification system is the same that is used in the SORA experiments.}
		\label{fig:stuart-track-types}
	\end{center}
\end{figure}

\subsection{Organic Solar Cell Methods}
\label{sec:Solar-Cell-Methods}


\subsubsection{Fabrication}

We began by developing the fabrication procedure for each layer in the stack. To achieve thin films from liquid solutions we use spin coating is for the EBL, ETL, and active layer. The liquid solution is deposited onto the surface of a substrate that is spun at a high RPM, flinging the solution out to the edges through centripetal forces and resulting a thin, ideally uniform, layer. After spinning the substrate is moved into the a hot plate where any remaining solvent is evaporated and a solid thin film remains. The top electrode is deposited by means of sputter coating. FTO slides, 0.5-1\% PEDOT:PSS in H$_2$O solution, and Pt sources were all used as purchased from suppliers without further modification. \\
	
	The EBL titania layer was synthesized through a sol gel process resulting in an amorphous phase, which when applied during spin coating and further annealed creates a dense noncrystalline layer. The sol gel titania solution was provided by Lily Schafer and prepared prior to our arrival at the lab. titania was held in the fridge when not in use.\\
	
	The active layer was prepared in a 1:1 ratio between P3HT and PCBM as suggested by literature and previous undergraduate UH lab procedures for "Excitonic photovoltaics". \SI{12.5}{\milli\gram} of PH3T was diluted in \SI{0.5}{\milli\liter} of chlorobenzene (CB) and \SI{12.5}{\milli\gram} of PCBM was also diluted in \SI{0.5}{\milli\liter} of CB. Both solutions were ultra sonicated before being combined into one solution and further sonicated to ensure an even mixing of P3HT and PCBM. We attempted to use a small magnetic stir bar to aggitate and mix the solution but this proved ineffective due to the small size of the stir bar and the height of the vessel containing our solution. Any active layer solution not immediately used is held in the desiccator. \\
	
	We will outline our typical fabrication process below without detailing any specific run:\\
	
	FTO slides are first cleaned with a chemical soap scrub and then with a deionized water, isoproponal, and acetone ultrasonic bath\textbf{<<FINISH THIS [model?] >>} successively before being dried under nitrogen and heated in the oven at \SI{150}{\celsius} for an hour. Clean slides were then taken to the spin coater for application of the titania EBL. The entire slide is loaded onto the spin coater and [<< FINISH THIS: how many? >>] mL were deposited from the autopipette before spinning at [RPM? seconds?] to achieve a uniform hazy film. The quality and thickness of  film was based on appearances as we did not have access to a profilometer. Immediately Following the film application a chemwipe or razorblade was used to remove a portion of the film from one side and expose the FTO. Each slide was then taken to the oven to anneal at \SI{550}{\celsius} for 5 hours, then cool back to room temperature over 12 hours. \\
	
	After annealing the slides were taken back to the spin coater. \SI{75}{\micro\liter} of active solution was deposited at 100 RPM for 5 second before ramping up to 1000 RPM for 25 seconds. Each slide was set on a hot plate at \SI{150}{\celsius} for 10 minutes before returning to the spin coater. \SI{200}{\micro\liter} PEDOT:PSS was deposited by drop casting before spin coating at 500 RPM for 5 seconds then ramped up to 5000 RPM for 25 seconds. Slides were then returned to the hot plate for 10 minutes. Each slide was then covered by a mask and set into the Leica EM SDC050 sputter coater that was loaded with a platinum source before being brought to a vacuum level ($<$\SI{10e-2}{\milli\bar}). The high voltage was then turned on and allowed to sputter for 150 seconds. \\
	
	Cells were analyze in the lab using CH instrument 600E model potentiostat along with 
