\subsection{Radiation Discussion}
\label{sec:Radiation-Discussion}

\subsubsection{Comsic Radiation}
As can be seen by comparing Figures \ref{subfig:minipix-counts} and \ref{subfig:fitpix-counts}, the MiniPIX and FITPix recorded a similar number of counts throughout the duration of flight despite the differing energy thresholds for the devices.
Both the trend and values of the 2019 counts almost exactly match that of the 2018 SORA mission, which is shown in Figure \ref{subfig:2018-minipix-counts}.
The MiniPIX and FITPix dose measurements are slightly different in value due to the difference in energy threshold between the two detectors, but the plots follow the same trend.
Interestingly, the 2019 MiniPIX data has a small but noticeably higher average dose than the 2018 MiniPIX data.
The average dose rate for the 2019 data is \SI{0.578}{\micro\gray\per\second} while the average dose rate over the same period of time for the 2018 MiniPIX data is \SI{0.499}{\micro\gray\per\second}.
There is a 15.8\% increase from 2018 to 2019.
While this is not a substantial increase, this increase can likely be attributed to the change in the device's environment.
Further investigation through future experiments is certainly needed in order to make a definitive claim.

The MiniPIX LET histogram possesses a remarkably similar shape to the LET histogram of the 2018 mission data, which can be seen in Figure \ref{fig:2018-minipix-let}.
The primary difference between the two data sets is the total counts, which is understandable considering the 2018 mission had a longer duration than the 2019 mission.
The density plots for both the MiniPIX and the FITPix LET spectra clearly show that the majority of LET values lie within the \SIrange{100}{1000}{\kilo\eV\per\micro\meter} range.

Further investigations into the effectiveness of the ISS module would certainly be worthwhile.
As previously mentioned, developing an accurate and low-cost radiation dosimeter is of increasing importance with the increasing presence of humans in space.
While computational models are a powerful tool, they are currently not well-suited for monitoring the well being of astronauts and their exposure to radiation.
Schwadron et. al. \cite{Schwadron-Update} from the CRaTER experiment found that the observed dose rates in the interplanetary environment exceed computational predictions by about 10\%.

Understanding and monitoring thermal neutrons is important for the safety of astronauts and airline passengers.
This idea was mentioned in the 2018 SORA mission report, which attempted to monitor neutrons with the use of a scintillator.
The scintillator used in that experiment did not yield any results, so further investigations into determining an effective scintillating material to use with a Timepix device is worthwhile.
Furthermore, monitoring the solar cycle with a Timpix dosimeter would be an interesting application and a good testbed of such a device.
Hathaway has shown a direct relationship between the average neutron counts and the development of the solar cycle \cite{Hathaway-Solar-Cycle}.

%The device configuration was not ideal since placing the MiniPIX in the control container would prove to be a more effective experimental control.
%The reason behind using the FITPix as the control is that the FITPix was not the property of our team and was borrowed from a different lab.
%In order to ensure the safety of the FITPix, it was placed inside the container which was known to be effective at keeping the device safe.


\subsubsection{Organic Solar Cells}

	With no prior experience in fabrication of organic semiconductors, The OPV program proved to be a greater, but rewarding, challenge than predicted. The radiation group gained indispensable experience in theoretical and experimental techniques of organic photovoltaics, ranging from the basic properties of conjugated polymers to synthesis of sol-gels and current-voltage profiling. Beyond the fundamental work with OPVs, we also considered the energy demands for future space missions. With the rapid rise of private space launch companies and the plans for trips to mars and beyond, lightweight and flexible solar panels with high specific power are most desirable to be developed. In situ manufactorability is one of the great advantages that OPVs offer to the space energy environment, where roll-to-roll processing can be conducted easily. 
A few days prior to the team's departure to New Mexico, several difficulties cropped up regarding the organic solar cell fabrication.
The fabrication lab was experiencing issues with the quality of the materials, and the cells that were produced could not meet the usual efficiencies of cells produced in the lab.
Furthermore, the current-reading circuit could not perform as intended.
The circuit itself was properly designed to account for the 24 cells that were originally planned to be used, but there was an incompatibility between the multiplexers and the operational amplifiers used in the circuit.
As a result of both the cell fabrication difficulties and the circuit incompatibility, the number of cells was reduced.
The current reading circuit was connected directly to the Arudino MEGA since the multiplexers needed to be removed.
Three cells were used for flight since the Arduino MEGA could only accommodate six direct connections to its analog pins (two connections per cell).
The reason for this incompatibility is still unknown, but it likely has to do with the internals of the chips.
We believe that continuded in
