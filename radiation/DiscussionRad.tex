\subsection{Radiation Discussion}
\label{sec:Radiation-Discussion}

\subsubsection{Comsic Radiation}
As can be seen by comparing Figures \ref{subfig:minipix-counts} and \ref{subfig:fitpix-counts}, the MiniPIX and FITPix recorded a similar number of counts throughout the duration of flight despite the differing energy thresholds for the devices.
Both the trend and values of the 2019 counts almost exactly match that of the 2018 SORA mission, which is shown in Figure \ref{fig:2018-minipix-counts}.
The MiniPIX and FITPix dose measurements are slightly different in value due to the difference in energy threshold between the two detectors, but the plots follow the same trend.
Interestingly, the 2019 MiniPIX data has a small but noticeably higher average dose than the 2018 MiniPIX data.
The average dose rate for the 2019 data is \SI{0.578}{\micro\gray\per\second} while the average for the 2018 MiniPIX data is \SI{0.499}{\micro\gray\per\second}.
There is a 15.8\% increase from 2018 to 2019.
While this is not a substantial increase, this increase can likely be attributed to the change in the device's environment.
Further investigation is certainly needed in order to make a definitive claim.

The MiniPIX LET histogram possesses a remarkably similar shape to the LET histogram of the 2018 mission data, which can be seen in Figure \ref{fig:2018-minipix-let}.
The primary difference between the two data sets is the total counts, which is understandable considering the 2018 mission had a longer duration than the 2019 mission.

The density plots for both the MiniPIX and the FITPix clearly show that the majority of LET values lie within the \SIrange{100}{1000}{\kilo\eV\per\micro\meter} range.



%The device configuration was not ideal since placing the MiniPIX in the control container would prove to be a more effective experimental control.
%The reason behind using the FITPix as the control is that the FITPix was not the property of our team and was borrowed from a different lab.
%In order to ensure the safety of the FITPix, it was placed inside the container which was known to be effective at keeping the device safe.
