\section{Results}
\label{sec:Results}

The SORA 3 payload recorded a total of 13859 data packets during flight.
The approximately 15 minutes gap seen in all data plots is due to the down time that the HASP administration ordered in response to the faulty battery sensor.
Fortunately, this gap in data did not cause any issues.
One important incident to note that occurred midflight was the overheating of the MiniPIX device.
This issue occurred when the device reached about \SI{90}{\celsius} at approximately 15:36:09 on September 5th.
Once this happened, the MiniPIX shutdown and the payload halted functionality.
The MiniPIX overheating was a direct result of the MiniPIX not having proper heat sinks.
The MiniPIX was directly connected to the aluminum chassis of the ISS module to use as a heat sink, but this was not effective enough.
Had the astrobiology system functioned properly a power cycle command would have been sent to the payload, and the entire system would have successfully rebooted.
Upon rebooting, the astrobiology deployment system would have retracted, however, this would have potentially caused the lid of the astrobiology system the become detached from the payload thus creating a loose projectile that would have fallen off the balloon gondola.
The team did not want to risk this happening, so the payload was not power cycled.
This resulted in a lack of data collection for the remainder of flight.

All systems withstood the landing impact.
The payload was returned to UH with no sustained damage, and every component was functional.
