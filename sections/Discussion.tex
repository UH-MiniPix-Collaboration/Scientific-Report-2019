\section{Discussion}
\label{sec:Discussion}

This is the discussion.

We opted to use a current to voltage converter, also known as a transimpedance amplifier, in order to collect data necessary to characterize the organic solar cells.
This was used primarily because it is a relatively simple circuit which was already known of that would not add any additional resistance in the solar cell's current path, thus not reducing the current through the cell.
The preferred method would be to utilize a proper current shunt monitor integrated circuit (IC) in a high-side sensing arrangement.
This would allow for more responsiveness to measuring current flowing through the solar cell.
The method of a shunt monitor works by measuring the current through a shunt resistor.
Since the shunt monitor is comprised of a match differential amplifier with high input impedance the shunt resistance seen by the solar cell would be practically zero and would also not alter the current flowing through the cell.

To collect readings of the voltages produced from the solar cell currents we opted to use the Arduino MEGA analogRead() capability which is able to read voltages up to 5V.
It is not recommended to read negative voltages but since we were reading voltages of very small currents the Arduino was not in danger of being destroyed.
The issue with using an Arduino is that it has an analog to digital converter (ADC) resolution of 10 bits at a maximum voltage reading of 5v which equates to an error of +/-4.89mV.
There is the Arduino Due which is capable of 12bit ADC resolution which would equate to an error of +/-1.22mV and would give a 25\% greater precision of data collection.
There are 16bit ADC's available as packaged IC's but they cane become cost prohibitive especially when collecting data from a large array of solar cells.