\section{Mission Objectives and Background}
\label{sec:Introduction}
\newcommand{\indentitem}{\setlength\itemindent{25pt}}%%%indent costume command

The Stratospheric Organism and Radiation Analyzer (SORA) 3 is the third generation of a family of experiments.
The overarching goal shared by each of the experiments is to collect stratospheric extremophiles and to characterize the radiation environment between the Earth's surface to the HASP float altitude ($\sim$\SI{35}{\kilo\meter}).
SORA 3 had the following scientific objectives:

\vspace{0.5cm}
\noindent \textbf {Primary Scientific Objectives:}
\begin{enumerate}
\item Capture microorganisms in the upper atmosphere at altitudes of approximately \SIrange{30}{35}{\kilo\meter} using a method not previously used by the UH HASP team. 
\item Study the cosmic and terrestrial radiation to which extremophiles and astronauts are exposed.
\item Observe the performance of organic solar cells in a near-space environment.
\end{enumerate}

\vspace{0.5cm}
\noindent \textbf {Secondary Scientific Objectives:}
\begin{enumerate}
\item Test the newly developed astrobiology hardware during flight and establish a more reliable method for collecting microbes in extreme environments at high-altitudes.
\item Establish a methodology which allows two or more Medipix devices to be used simultaneously.
\item Study and test our organic solar cell fabrication methods in a highly irradiated environment.
\end{enumerate}

\vspace{0.5cm}
\noindent \textbf {Engineering Objectives:}
\begin{enumerate}
\item Develop a new astrobiology collection mechanism that is favorable at high altitude.
\item Construct a structure resembling an ISS module as accurately as possible.
\item Analyze radiation data in real time and downlink relevant information.
\item Develop active layers for organic solar cells which can withstand the stratospheric environment.
\end{enumerate}

