\section{SORA 3 Payload Design}
\label{sec:Design}

\subsection{Payload Structure}
One major design goal of the SORA 3 mission was to make the overall layout of the payload more modular such that each subsystem was not dependent on the presence of other subsystems.
This was a major change from previous missions in which systems and components were not organized leading to massive disorganization across the payload.
The modular structure not only organized the systems, but it made it such that the now independent systems could function in the absence of other systems, thus leading to continued functionality of the payload in the event of an in situ failure.

The modular organization was accomplished by compartmentalizing each subsystem into its own structure.
The payload was broken up into four main substructures: the astrobiology system, the two radiation containers, and the electronics box.
The electronics box served to join all systems together, and the other subsystems could easily be added or removed to the electronics box.
The electronics box was made from an easy-to-form PVC/acrylic, which has been proven to work well for previous missions as it can withstand the extreme environment of the stratosphere and is easily machinable.
The construction and materials of the astrobiology box and the radiation containers are discussed in their respective sections.

% Include image of the payload layout
% Talk about physical construction and layout of the payload


% Talk about the hardware (PSU, RPi, etc.)
\subsection{Hardware and Electronics}
The SORA 3 payload used a WinSystems PPM-DC-ATX-P power supply unit (PSU).
This unit has several properties that are very attractive for applications such as this. 
To power the flight computer and hardware we received the 30VDC supply from HASP which we fed to a PPM-DC-ATX-P power supply that supplied our electronics with \SI{+5}{volt}, and our current-to-voltage operational amplifiers with \SI{+12}{volt} and \SI{-12}{volt}.
It has an input range of \SIrange{10}{50}{\volt}, so the unit is capable of withstanding any variation in voltage caused by HASP's depleting batteries.
The pins used for the SORA 3 payload are marked in Figure \ref{fig:PSU-Outputs}. << FINISH THIS: add figure >>
Additionally, this PSU is rated to operate from \SIrange{-40}{85}{\celsius}, which is within the environmental variation of temperatures in the stratosphere.
More technical details can be found on its datasheet at Ref. \cite{WinSystems-PSU}

A Raspberry Pi was chosen as the main component of the flight computer as we needed a platform that was capable of storing an SQL database for the large volume of data we had planned to catalogue from the solar cells. An SQL database was found to be necessary due to the serial down-link limit imposed by HASP. We also opted to utilize an Arduino MEGA in order to control the majority of the sensors such as the thermistors for temperature readings, multiple photodiodes for capturing light intensity at the solar cells, a pressure sensor for monitoring altitude, and the array of current-to-voltage converters for characterizing the solar cells. Additionally, the Arduino MEGA sent control signals to an Arduino Nano, a very small footprint microcontroller, which controlled an actuator and a DC stepper motor that were required to operate the mechanical system to facilitate the Astrobiology experiment. 