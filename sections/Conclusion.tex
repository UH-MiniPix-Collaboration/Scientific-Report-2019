\section{Conclusion}
\label{sec:Conclusion}
%%%
%General Guidlines for a conclusion:
%•States whether the purpose was accomplished, and/or hypothesis was supported
%•Backs this up by referring to results
%•Reports final result with uncertainty 
%•Answers any questions posed in the lab manual
%•Addresses any pertinent issues; summarizes discussion, possible sources of error, possible experimental improvements, what has been learned, etc
%
%%%%%this is the conclusion from last year, just for reference.  MAKE SURE TO CITE OUR OWN WORK!!!!!!!!!!!!!!!!
%%RESU
%RESU proved to be efficient and robust.  RESU maintained current below 1.5 $A$ at 30 $V$.  It monitored the environment for the whole duration of the flight flawlessly.  Most sensors operated without a hitch, but a couple of sensors had some issues. The analog temperature sensor used for the electronic pump was found to be reading erroneously, while the magnetometer and orientation sensor recorded noise generated by the electric pump onboard.  Overall, the mission was a testbed for this prototype and allowed us to improve on RESU for future flights.
%
%%Astrobiology
Although DNA was detectable in our samples, we were unable to advance to the next phase in the process of sequencing. We are hopeful that alterations to the DNA extraction protocols will allow for more definitive results in the future. Additionally, a new collection assembly designed to maximize the experimental sample volume is one of the main objectives of next year’s flight. Regarding the radiation, this year's mission provided a foundation for future flights. The noticeable difference in overall particle counts may be correlated with the addition of the plastic scintillator. The MiniPIX system can be expanded further by the addition of more MiniPIX devices, which would yield a larger volume of data. This would allow stronger correlations to built with regards to this year's exploration of neutrons. There is much to be learned from living cells that can survie under the float conditions our payload was subjected to at altitude. Genomic, proteomic, and metabolomics comparisons between the high-altitude atmospheric samples and near/at surface samples may yield useful insight into organic radiation shielding and adaptations for survival in extreme conditions. Further missions would help expand on discovering more about these organisms.
%
%%Radiation
%The particular photodiode we chose for this flight did not have the characteristics to endure the intense exposure of UV light. Nonetheless, the apparatus and ND filters allowed the diodes to operate continuously without saturation under the extreme solar UV irradiance at high altitude. The average maximum irradiance recorded between the three UV photodiodes was \SI{34.8}{\watt\per\meter\squared}. The MiniPIX survived the mission with no issues from the harsh environment in the stratosphere or the crash. The onboard heatsink kept the MiniPIX within working temperatures. Overall, the MiniPIX gathered a wealth of information regarding surrounding radiation, with a peak dosage rate of about \SI{4.2}{\micro\gray\per\hour} at ascent and about \SI{3}{\micro\gray\per\hour} during float.
%
%%Dissemination
%The work performed throughout this project has already begun to be shared around the greater Houston area, starting with presentations at the University of Houston Undergraduate Research Day~\cite{SamURD}~\cite{StevenURD}, expanding to the local Houston community~\citep{StevenSchoolPres}~\cite{Fre}, and even to the APS Texas Section~\cite{SamAPS} in Dallas, TX.  The final results will soon be published in peer reviewed journals, with the articles in preparation now.  Additionally, the team is seeking to patent the different systems and computer systems.  RESU was entered into a competition with MIT for the MIT-Lemelson Award~\cite{MIT} in the "Use it!" catergory. %here it is
