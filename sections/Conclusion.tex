\section{Conclusion}
\label{sec:Conclusion}

Overall, the SORA 3 payload did not succeed.
The SORA 3 experiment received a major overhaul in comparison to the previous two iterations of SORA, and this resulted in some new issues that arose during flight.
The astrobiology experiment failed due to deformation in the lid of the astrobiology box.
The tight tolerance between the bottom of the filter flaps and the top rim of the clean box resulted in the lid getting stuck on the side of the box, and the experiment to not function properly.
Had this gap been a couple of millimeters larger, this issue would not have occurred, and the astrobiology box lid would have been able to rotate freely.
The failure of the astrobiology experiment had a snowball effect, which resulted in an overall shutdown of the payload.
The MiniPIX device overheated, and this resulted in a halt of data collection for the entire payload.
If the astrobiology system was functioning properly, the payload could have been rebooted in order to restart the data collection system, and the payload could have functioned normally.

Despite the issues faced during the SORA 3 experiment, the MiniPIX and FITPix devices were able to return meaningful data that can be compared to previous and future radiation dosimetry flights.
Additionally, the methodlogy tried by the SORA 3 payload has resulted in a wealth of knowledge for future attempts.
Slight improvements can be made to the astrobiology and radiation systems to ensure proper and continued functionality of the systems for the duration of a typical balloon flight.
Furthermore, the introduction of the organic solar cell experiment led to a clear definition of what works for a solar cell experiment in this context.

Most importantly, the SORA 3 experiment offered real-world, hands-on experience to over a dozen undergraduate students.
Every participant gained in-depth knowledge in their particular fields of interest and had the opportunity to make a contribution to a project that lies within that field.
Had the team been given the opportunity to continue this work, the next iteration of the SORA experiment would have been able to utilize the information gained through the trials of the SORA 3 experiment and huge improvements could have been made.


