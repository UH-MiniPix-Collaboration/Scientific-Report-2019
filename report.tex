\documentclass[aps,superscriptaddress,floatfix,nofootinbib,showpacs,amsmath,amssymb,altaffilletter,floatfix,onecolumn]{revtex4-1}

\input{def.tex}
\usepackage{fancyhdr}
\usepackage{enumitem}
\usepackage{siunitx}
\usepackage{graphicx}
\usepackage{subfigure}
\usepackage{wrapfig}
\usepackage[title]{appendix}
\usepackage{float}
\usepackage{tabularx}

\fancyhf{}
\renewcommand{\headrulewidth}{0pt}
\rfoot{\thepage}
\pagestyle{fancy}
\renewcommand{\thepage}{}
\renewcommand{\thepage}{\arabic{page}}
\renewcommand\thesection{\arabic{section}}
\renewcommand\thesubsection{\thesection.\arabic{subsection}}
\renewcommand\thesubsubsection{\thesubsection.\arabic{subsubsection}}
\newcolumntype{Y}{>{\centering\arraybackslash}X}
\parskip = 6pt %changes spacing between paragraphs
%\nolinenumbers
\makeatletter
\def\p@subsection{}
\makeatother
\makeatletter
\def\p@subsubsection{}
\makeatother

\begin{document}
\title{SORA 3: Stratospheric Organism and Radiation Analyzer}

\begin{abstract}
\begin{center}
{\bf Abstract}
\end{center}
\frenchspacing

The SORA 3 experiment sought to continue the work performed by the previous two SORA iterations.
The primary goal of the SORA 3 experiment was to use the foundation that was developed over the previous two years to improve the astrobiology collection mechanism, expand the existing radiation system to account for two Timepix devices, and add a new organic solar cell study. 
The new payload featured an overhauled astrobiology system which utilized mechanical rotation to sample the stratospheric environment.
Using the software developed for the 2018 SORA mission, a FITPix device was added to the payload.
The new device allowed the previously used MiniPIX device to be housed inside a mock-up International Space Station (ISS) module which sought to simulate the environment inside the actual ISS.
The software was modified to account for the additional Timepix device so that the two devices would simultaneously record data.
Lastly, the organic solar cell experiment aimed to expose such cells to the stratosphere and observe the structural degradation and performance change to the cells.
Overall, the SORA 3 experiment did not succeed as each experiment had its own difficulties and points of failure.
The material of the astrobiology container experienced deformation which prohibited the mechanical arm from spinning, the Timepix devices collected data for several hours but stopped and could not be rebooted due to the disfunctional astrobiology motor, and the solar cell fabrication lab had issues with material quality during the days leading up to flight. 
Despite these failures, the design and methodology of the payload provided valuable knowledge and can serve as a stepping stone for future iterations of the SORA experiments. 

\newpage %Breaks page for the Table of Contents.
\end{abstract}
\newcommand{\Physics}{College of Natural Sciences and Mathematics, Department of Physics, University of Houston, Houston, TX, 77204, USA}
\newcommand{\CS}{College of Natural Sciences and Mathematics, Department of Computer Science, University of Houston, Houston, TX, 77204, USA}
\newcommand{\Biology}{College of Natural Sciences and Mathematics, Department of Biology, University of Houston, Houston, TX, 77204, USA}
\newcommand{\BME}{Cullen College of Engineering, Department of Biomedical Engineering, University of Houston, Houston, TX, 77204, USA}
\newcommand{\Chemical}{Cullen College of Engineering, Department of Chemical and Biomolecular Engineering, University of Houston, Houston, TX, 77204, USA}
\newcommand{\Electrical}{Cullen College of Engineering, Department of Electrical Engineering, University of Houston, Houston, TX, 77204, USA}
\newcommand{\Mechanical}{Cullen College of Engineering, Department of Mechanical Engineering, University of Houston, Houston, TX, 77204, USA}
\newcommand{\CSU}{Walter Scott Jr. College of Engineering, School of Advanced Materials Discovery, Colorado State University, Fort Collins, CO, 80521}

\author{R.~B.~Masek}\affiliation{\Physics}
\author{T.~D.~Hill}\affiliation{\CSU}
\author{D.~W.~Howard}\affiliation{\Electrical}
\author{J.~Patel}\affiliation{\Physics}
\author{J.~Alvarado}\affiliation{\Physics}
\author{C.~Amaya}\affiliation{\CS}
\author{A.~Boggs}\affiliation{\Physics}
\author{C.~Bush}\affiliation{\Biology}
\author{A.~Carpy}\affiliation{\Physics}
\author{K.~Fleming}\affiliation{\Chemical}
\author{E.~Humble}\affiliation{\Physics}
\author{S.~Ngiang}\affiliation{\Mechanical}
\author{K.~Ngo}\affiliation{\Physics}
\author{H.~Trong}\affiliation{\Mechanical}
\author{A.~Vega}\affiliation{\Mechanical}
\author{S.~George}\affiliation{\Physics}
\author{I.~Wilson}\affiliation{\Biology}
\author{D.~Pattison}\affiliation{\Biology}
\author{P.~Gunaratne}\affiliation{\Biology}
\author{A.~L.~Renshaw}\affiliation{\Physics}
\author{O.~K.~Varghese}\affiliation{\Physics}

\setlength{\parindent}{1em}
\setdefaultleftmargin{1em}{1em}{}{}{}{}
%---
\setcounter{page}{0}\thispagestyle{empty}
%---
\maketitle
\onecolumngrid
\setcounter{tocdepth}{2}
\setcounter{page}{0}\thispagestyle{empty}
\tableofcontents
\setcounter{page}{0}\thispagestyle{empty}
\newpage
\onecolumngrid

%Section: Introduction
\subimport{sections/}{Introduction.tex}
\subimport{astrobio/}{IntroAstro.tex}
\subimport{radiation/}{IntroRad.tex}

%Section: Payload Description
\subimport{sections/}{Design.tex}
\subimport{astrobio/}{DesignAstro.tex}
\subimport{radiation/}{DesignRad.tex}
\subimport{sections/}{Telemetry.tex}

%Section: Methods
\subimport{sections/}{Methods.tex}
\subimport{astrobio/}{MethodsAstro.tex}
\subimport{radiation/}{MethodsRad.tex}

%Results and Analysis
\subimport{sections/}{Results.tex} 
\subimport{astrobio/}{ResultsAstro.tex} 
\subimport{radiation/}{ResultsRad.tex} 

%Discussion
\subimport{sections/}{Discussion.tex}
\subimport{astrobio/}{DiscussionAstro.tex}
\subimport{radiation/}{DiscussionRad.tex}

%Conclusion
\subimport{sections/}{Conclusion.tex} 
\newpage

%Appendix
\subimport{sections/}{Appendix.tex}
\newpage

%References
\subimport{sections/}{Bib.tex}

\clearpage
\bibliographystyle{SORA}
\end{document}
